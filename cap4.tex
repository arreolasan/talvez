\chapter{Conclusiones}

\section{Conclusiones}
El estudiante debe escribir las principales conclusiones del proyecto en base a los objetivos y actividades programadas,
contrastando con lo realmente realizado y resultados alcanzados.
Realizar recomendaciones pertinentes respecto al uso de los resultados obtenidos,
insuficiencias observadas y propuestas en el proceso desarrollado.

Especificar cual fue el alcance del trabajo.
Esta parte del trabajo esta basada en el an�lisis y evaluaci�n de cada uno de los objetivos que se plantearon en el estudio.
Es importante cuidar de no establecer conclusiones que no est�n respaldadas por resultados.

\section{Trabajo a futuro}
De ser posible, se plantearan mejoras tendientes a obtener mejores resultados,
las cuales justificar�an un trabajo posterior que pudiese ser motivo de otra residencia. 

\section{Competencias desarrolladas}
Para cerrar la secci�n, el estudiante debe describir las competencias gen�ricas y especificas aplicadas y las adquiridas durante el desarrollo del proyecto de Residencia Profesional.







%%%%%%%%%%%%%%%%
% si el contenido del capitulo termina en hoja PAR agregar \sigpag,
% si el contenido del capitulo termina en hoja IMPAR poner %\sigpag (que es el equivanete a quitar la instrucci�n)
\sigpag