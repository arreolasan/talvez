%%%%%%%%%%%%%%%% preambulo
\documentclass[carta]{memoria}
\usepackage{anysize}
\usepackage{doc}
\usepackage[latin1]{inputenc} 
\usepackage[pdftex]{color,graphicx}
\marginsize{3cm}{2.5cm}{2cm}{1.5cm}

%%%%%%%%%%%%%%%% datos usados en el documento
\newcommand{\NombreT}{``Nombre del proyecto''}
\newcommand{\Alumno}{Nombre ApellidoP ApellidoM}
\newcommand{\noControl}{12345678}
\newcommand{\FechaEntrega}{abril 2023}
\newcommand{\AsesorI}{Dr. Nombre y Apellidos}
\newcommand{\AsesorE}{Ing. Nombre y Apellidos}

%%%%%%%%%%%%%%%% palabras que corte mal el latex (ver notas de: LEEME informacion.txt)
\hyphenation{re-pre-sen-tar-lo a-pro-xi-ma-da-men-te}

%%%%%%%%%%%%%%%% comienza el documento, portada
\begin{document}
\pagestyle{empty}
\portada
\frontmatter
\pagestyle{plain}

%%%%%%%%%%%%%%%% Hoja de agradecimientos
%pocos documentos anexan esta parte, sugiero solo agregar en caso de:
%   pr�stamo de equipo especial
%   o beca por parte de alguna instituci�n:
%\addcontentsline{toc}{section}{Agradecimientos}
%.\par \vspace{3cm}
%\begin{flushright}
%\Large\textsc{Agradecimientos}
%\end{flushright}
%Mi agradecimiento a... por el prestamo del laboratorio ...\\
%
%Este trabajo fue financiado por CONACYT con la beca de manuntenci�n No. XXX-cb (Enero 2022- Junio 2022).
%\sigpag

%%%%%%%%%%%%%%%% Resumen (ver notas de: LEEME informacion.txt)
.\par \vspace{3cm}
\begin{flushright}
\Large\textsc{Resumen}
\end{flushright}
\section*{\NombreT}
\addcontentsline{toc}{section}{Resumen}
Actualmente, ...

Por lo tanto se vio la necesidad de ...

Lo que se realizo fue ....

Este proyecto ...\\

\textbf{Palabras Clave:}
Dise�o Asistido por Computadora (CAD), Microcontrolador, Algoritmo Evolutivo, etc. \\

\footnotesize{\copyright ~TecNM/IT de Tijuana, \FechaEntrega . Derechos Reservados. El autor: \Alumno ~otorga al TecNM/IT de Tijuana el permiso de reproducir y distribuir copias de este reporte en su totalidad o en partes.}\\
\sigpag

%%%%%%%%%%%%%%%% �ndices
\addcontentsline{toc}{section}{�ndice de Contenido}
\tableofcontents \par
%\sigpag   %si el contenido termina en hoja PAR agregar esta instrucci�n
\addcontentsline{toc}{section}{�ndice de Figuras}
\listoffigures \par
\sigpag   %si el contenido termina en hoja PAR agregar esta instrucci�n
\addcontentsline{toc}{section}{�ndice de Tablas}
\listoftables \par
\sigpag   %si el contenido termina en hoja PAR agregar esta instrucci�n

%%%%%%%%%%%%%%%%% Glosario
%%pocos documentos anexan esta parte,
%%sugiero solo agregar en caso de tener t�rminos diferentes al �rea de electr�nica o biom�dica
%\newpage
%\addcontentsline{toc}{section}{Glosario}
%\begin{flushright}
%\Large\textsc{Glosario}
%\end{flushright}
%T�rmino: definici�n
%\sigpag   %si el contenido termina en hoja PAR agregar esta instrucci�n

%%%%%%%%%%%%%%%% cuerpo del documento
\mainmatter
\setlength\parskip{6mm}
\pagestyle{headings}
{
\chapter{Introducci�n}

En este capuitulo se describen los antecedentes y el planteamiento del problema (contenido del anteproyecto).
Se finaliza la secci�n describiendo la organizaci�n del documento.

\section{Descripci�n de la Instituci�n o empresa}

Se describe brevemente la Instituci�n o empresa en la cual el estudiante realiz� su
proyecto de Residencia Profesional,
as� como su puesto y �rea de trabajo. Por ejemplo:

El Instituto Tecnol�gico de Tijuana (ITT) fue fundado el 17 de septiembre de 1971
como una instituci�n de educaci�n a nivel superior y medio superior,
siendo una de las primeras instituciones en ofrecer educaci�n superior tecnol�gica en el estado.
Sus primeras carreras profesionales fueron las de Ingenier�a Electromec�nica en Producci�n y Licenciatura en Relaciones Industriales.
Actualmente su funci�n es de brindar servicios de educaci�n superior a nivel licenciatura,
maestr�a y doctorado sobre la base de las necesidades de talento humano en la regi�n.

El ITT representa la principal oferta de educaci�n tecnol�gica en el estado de Baja California,
en el primer semestre de 2018 atendi� a una matr�cula de un poco m�s de 7 mil estudiantes de licenciatura, maestr�a y doctorado \cite{webMCI}.

%Cuando se utilice referenciar a las fuentes de informaci�n, se usa \cite{}
%el cual cita usando en el formato IEEE mostrado en referencias,
%latex da un n�mero automaticamente a las referencias, como se vio arriba al usar \cite{webMCI}

En la Figura \ref{organ1} se aprecia el organigrama donde se lleva al cabo la residencia profesional.
Es el organigrama que utilizaban mis alumnos.
%Generalmente la palabra figura en el texto va con mayuscula: Figura
%y cada vez que se coloque una imagen, esta debe se mencionarse en el texto, como en la oracion anterior

\begin{figure}[hbt!]
\centering
\includegraphics[width=0.6\textwidth]{figuras/organigrama}
\caption{Organigrama del Posgrado en Ciencias de la Ingenier�a, en 2019.}
\label{organ1}
\end{figure}

\section{Objetivos de la Residencia}
Los objetivos son los prop�sitos del estudio y expresan el fin que pretende alcanzarse y,
por lo tanto,
todo el desarrollo del trabajo  de investigaci�n se orientar� a lograr estos objetivos.
No es necesario escribir pre�mbulos al momento de redactar los objetivos. 
En toda investigaci�n es necesario plantear 2 niveles en los objetivos: el general y el espec�fico.

\textbf{Objetivo General:} debe reflejar la esencia del planteamiento del problema y la idea expresada en el t�tulo del proyecto de investigaci�n.

\textbf{Objetivos Espec�ficos:} se desprenden del general y deben ser formulados de forma que est�n orientados al logro del objetivo general, es decir; que cada objetivo espec�fico est� dise�ado para lograr un aspecto de aqu�l, y todos en su conjunto.

%%%% A veces es necesario utilizar vi�etas, por ejemplo:
Como metas secundarias se tiene:
\begin{itemize}
	\item Proponer ...
	\item Desarrollar ...
	\item Encontrar nuevas ...
	\item Proponer ...
	\item Desarrollar ...
\end{itemize}


\section{Antecedentes y Justificaci�n}
Importancia y beneficios que se lograron con el desarrollo del proyecto respecto de la problem�tica planteada.
En esta secci�n se deber� explicar por qu� el proyecto es una opci�n factible.
Adem�s se deber� dejar claro cu�l es el alcance del proyecto dentro y fuera de la empresa, as� como las limitaciones a las que se sujeta el mismo.


%%%%%%%%%%%%%%%%
% si el cap�tulo termina en hoja IMPAR latex agrega una par en blanco,
% esto porque la impresion es por ambas hojas
% para no dejar hojas en blanco se suguiere usar el comando: \sigpag
% (ver notas de: LEEME informacion.txt)
\sigpag
\chapter{Marco Te�rico}

\section{Introducci�n}
El marco te�rico debe contener los fundamentos necesarios para llevar a cabo el desarrollo del proyecto.
Sin caer en un corta-pega y/o una extensi�n excesiva y dando el cr�dito que correspondiente a las fuentes consultadas.
Este cap�tulo puede ser particionado en dos o tres
cap�tulos si los asesores interno y externo deciden que es la mejor opci�n para organizaci�n y
presentaci�n del documento.

El marco te�rico describe las herramientas y t�cnicas de Ingenier�a (Industrial, Sistemas Computacionales, Electromec�nica, Mecatr�nica) que ser�n utilizadas para resolver la situaci�n problem�tica expresada en este trabajo de residencias profesionales.
Al marco te�rico le corresponde la funci�n de orientar y crear las bases te�ricas de la investigaci�n.
El marco te�rico no es un resumen  de las teor�as que se hayan escrito sobre el tema objeto de la investigaci�n.
Es una revisi�n de quienes est�n investigando o han investigado el tema y los planteamientos de estos autores y cu�les son los principales aspectos por ellos estudiados.

De estas herramientas se debe dar:
\begin{itemize}
	\item La definici�n
	\item Conceptos
	\item Historia
	\item Descripci�n de los m�todos, t�cnicas, a emplear en la resoluci�n del problema
	\item Actualidad que existe sobre el tema y resultados que se hallan obtenido, entre otras cosas
  \item Siempre citando las referencias donde se sustenta lo investigado, ej: ...como se ve en \cite{libr1,capi1,revi1}; sin embargo, en \cite{pagi1} se utiliz� ...\\
\end{itemize}
%Cuando se utilice referenciar a las fuentes de informaci�n, se usa \cite{}
%el cual cita usando en el formato IEEE mostrado en "referencias",
%latex da un n�mero automaticamente a las referencias, como se vio en el ultimo item de arriba

\section{Tema 1 investigado}
Los ... como por ejemplo ...

\subsection{Subtema 1 del Tema 1}
La parte primaria ... aqui va lo investigado

\subsection{Subtema 2 del Tema 1}
La parte secundaria ... aqui va lo investigado, y asi sucesivamente


\section{Figuras}
Si se utilizan figuras recordar referenciarlas en el texto,
he indicar una referencia en caso de que sean tomadas de alguna fuente, ejemplo:
TEXTO ... cuya pintura se muestra en la Figura \ref{dragon1} \cite{capi1}.

\begin{figure}[hbt!]
\centering
\includegraphics[width=0.6\textwidth]{figuras/ciruelo1}
\caption{Obra del artista Ciruelo.}
\label{dragon1}
\end{figure}

\section{Tema 2 investigado}
Los ... se definen como ...

\subsection{Subtema 1 del Tema 2}
La parte primaria ... aqui va lo investigado

\subsection{Subtema 2 del Tema 2}
La parte secundaria ... aqui va lo investigado, y asi sucesivamente

\section{Tablas}
Si se utilizan tablas recordar referenciarlas en el texto,
he indicar una referencia en caso de que sean tomadas de alguna fuente.
Al igual que la palabra Figura, la palabra Tabla tambi�n se acostumbra escribirla con mayuscula, ejemplo:
TEXTO ... en la Tabla \ref{Tuno} \cite{tesis1}.
Observe que las citas van antes del punto, otro ejemplo de colocar datos en Recuadros se ve en la Tabla \ref{TarjHD}.

\begin{table}[!htb]
\caption{Tabla a modo de ejemplo.}
\begin{center}
\begin{tabular}{|c|r|l|}							%|c|r|l| significa: linea-centrar-linea-derecha-linea-izquierda-linea
\hline
\multicolumn{3}{|c|}{uni tres columnas}\\ %para esta linea se cambia |c|r|l| por: |c|
\hline
t1&t2&t3\\														%el resto de las lineas sigen el formato |c|r|l|
\hline
0&0&0\\
00&00&00\\
000&000&000\\
0000&0000&0000\\
\hline
\end{tabular}
\end{center}
\label{Tuno}  %con este nombre se referencia en el texto: \ref{Tuno}
\end{table}

\begin{table}[!htb]
\caption{Caracteristicas de tarjeta de audio HD.}
\begin{center}
\begin{tabular}{|c|l|}
\hline
Rango&Caracteristicas\\
\hline
 &Flujo de I/O, compatibilidad con escalabilidad\\
0 - 14 &para el control, enlace (link) y codec dise�ada\\
 &para optimizar el rendimineto y caracteristicas\\
\hline
6K Hz y 192K Hz&Frecuencias de muestreo\\
\hline
xxxx&yyy\\
\hline
 &....\\
48 Mbps& ...\\
 &...\\
\hline
\end{tabular}
\end{center}
\label{TarjHD} %con este nombre se referencia en el texto: \ref{TarjHD}
\end{table}

%%%%%%%%%%%%%%%% termina el capitulo
%\sigpag
\chapter{Desarrollo y resultados}

\section{Metodo propuesto}
En este cap�tulo se describen las actividades realizadas para cumplir con los objetivos espec�ficos propuestos.
Se discuten los resultados obtenidos respecto al objetivo general ilustrando mediante diagramas;
tablas; modelos matem�ticos; teoremas; series de tiempo, planos y espacios de fase; secciones de c�digo fuente, im�genes de interfaces gr�ficas;
pseudoc�digos; im�genes de ayudas visuales; procedimientos; fotograf�as de prototipos, estaciones de trabajo, fixturas, etiquetas, etc.

Explicar de una manera clara como se utilizaron cada una de las herramientas de Ingenier�a (Industrial, Sistemas Computacionales, Electromec�nica, Mecatr�nica), descritas en el marco te�rico, para llevar a cabo los planteamientos a posibles soluciones de problema que se estudia.

\section{Instrumentos empleados}
De haberse dado se deber�n anexar:\\
Los formatos que se establecieron para un mejor control de las actividades que se desarrollan.\\
Planos, gr�ficas y/o diagramas que auxilien para una descripci�n clara de las propuestas hechas para la soluci�n del problema.\\
Los c�lculos realizados para la soluci�n del problema.

\section{Resultados}
Especificar cuales son los resultados obtenidos para aplicar las herramientas de Ingenier�a (Industrial, Sistemas Computacionales, Electromec�nica, Mecatr�nica).

El prototipo dise�ado se muestra en la figura ... etc.



%%%%%%%%%%%%%%%%
\sigpag
\chapter{Conclusiones}

\section{Conclusiones}
El estudiante debe escribir las principales conclusiones del proyecto en base a los objetivos y actividades programadas,
contrastando con lo realmente realizado y resultados alcanzados.
Realizar recomendaciones pertinentes respecto al uso de los resultados obtenidos,
insuficiencias observadas y propuestas en el proceso desarrollado.

Especificar cual fue el alcance del trabajo.
Esta parte del trabajo esta basada en el an�lisis y evaluaci�n de cada uno de los objetivos que se plantearon en el estudio.
Es importante cuidar de no establecer conclusiones que no est�n respaldadas por resultados.

\section{Trabajo a futuro}
De ser posible, se plantearan mejoras tendientes a obtener mejores resultados,
las cuales justificar�an un trabajo posterior que pudiese ser motivo de otra residencia. 

\section{Competencias desarrolladas}
Para cerrar la secci�n, el estudiante debe describir las competencias gen�ricas y especificas aplicadas y las adquiridas durante el desarrollo del proyecto de Residencia Profesional.







%%%%%%%%%%%%%%%%
% si el contenido del capitulo termina en hoja PAR agregar \sigpag,
% si el contenido del capitulo termina en hoja IMPAR poner %\sigpag (que es el equivanete a quitar la instrucci�n)
\sigpag

\begin{thebibliography}{99}
\addcontentsline{toc}{section}{Referencias}
%%%%%%%%%%%%%%%%
%Las referencias deben de ser escritas en un solo estilo bibliogr�fico para uniformizar su presentaci�n
%en el documento, como el IEEE para Electr�nica y el Vancouver para Biom�dica.
%
% las referencias utilizadas pueden ser de varios tipos, solo se colocaron aqu� los m�s utilizados.
% se coloca en \bibitem el nombre de referencia, para ser utilizado automaticamente en todo el documento
%
% en el documento se le llama a las referencias con el comando cite, ejemplo: \cite{libr1}
%
% cuando coloco la palabra autor siempre me refiero al formato:
% apellidoP-apellidoM iniciales de los nombres; ejemplo: Duarte-Villase�or M.A.


%%%%%%%%%%%%%%%% formato IEEE (electr�nica)

%libro
\bibitem{libr1}
autor1, autor2. nombre-del-libro. editorial, edici�n, lugar, a�o.

%capitulo de libro
\bibitem{capi1}
autor1, autor2, autor3. nombre-del-capitulo-del-libro, cap.X del nombre-del-libro. pp. X-X, nombre de los editores, editorial. edici�n, lugar, a�o.

%Tesis
\bibitem{tesis1}
autor. nombre-de-la-tesis. Tesis de Licenciatura-Maestria-Doctorado, Escuela-o-Facultad del Universidad-o-Instituto, lugar, a�o.

%pag web
\bibitem{pagi1}
autor1, autor2. titulo-pag [Online]. lugar, fecha del trabajo. Consultado el: d�a-mes-a�o.\\
http://www.pagina.com
%ejemplo:
\bibitem{webMCI}
Instituto Tecnol�gico de Tijuana pagina WEB oficial. Tijuana, B.C. Consultado el: 04 de abril de 2023.\\
https://www.tijuana.tecnm.mx/

%articulos de revista
\bibitem{revi1}
autor1, autor2, autor3. nombre-del-articulo, en nombre-de-la-revista. Vol.X, No.XX, mes-y-a�o, pp. yy-zz.

%articulos de conferencias
\bibitem{confe1}
autor1, autor2. nombre-de-la-conferencia, en nombre-del-evento. mes a�o, pp. yy-zz.


%%%%%%%%%%%%%%%% formato Vancouver (biom�dica)
%libro
\bibitem{libr2}
autor1, autor2. nombre-del-libro. editorial, edici�n, lugar, a�o.

%articulos de revista
\bibitem{revi2}
autor1, autor2, autor3. nombre-del-articulo. nombre-de-la-revista. a�o; Vol(No.): pInicial-pFinal.

%pag web
\bibitem{pagi2}
nombre [Internet]. fecha del copyright o �ltima actualizaci�n, si se desconoce, poner n.d.; [consultado fecha]. disponible en: http://www.pagina.com 


\sigpag
\end{thebibliography}
%\sigpag   %si el contenido termina en hoja PAR agregar esta instrucci�n

\appendix

%%%%%%%%%%%%%%%% NOTA IMPORTANTE
%en los apendices o anexos no se soporta acentos normales á é í ó ú ü
%la forma de colocar acentos es: \'a \'e \'i \'o \'u \"u 

%Los anexos deben citarse en donde sea pertinente a lo largo del documento.
%El anexo A es la Carta de presentación emitida por la Institución
%y autorización por parte de la empresa o Institución
%para que el estudiante desarrolle su proyecto de Residencia Profesional.

\chapter{Carta de presentaci\'on}

%Carta de presentaci\'on emitida por la Instituci\'on y autorizaci\'on por parte de la empresa o Instituci\'on para que el estudiante desarrolle su proyecto de Residencia Profesional.

\includegraphics[width=0.9\textwidth]{figuras/CartaFirma.pdf} \\


%%%%%%%%%%%%%%%%
% si el contenido del capitulo termina en hoja PAR agregar \sigpag,
% si el contenido del capitulo termina en hoja IMPAR poner %\sigpag (que es el equivanete a quitar la instrucción)
\sigpag

\chapter{nombre del anexo}

%%%%%%%%%%%%%%%% NOTA IMPORTANTE
%en los apendices o anexos no se soporta acentos normales á é í ó ú ü
%la forma de colocar acentos es: \'a \'e \'i \'o \'u \"u 

Productos acad\'emicos obtenidos.

Hojas de especificaciones o notas que no son transcendentales en el reporte.

M\'as cosas definir por el estudiante y los asesores interno y externo.


% quitar la hoja blanca despues del anexo (ver notas de tareas finales):
\sigpag

%%%%%%%%%%%%%%%% colocar los anexos necesarios....
%\input{Anexo3.tex}
%\input{Anexo4.tex}
\par}

\findoc
\end{document}