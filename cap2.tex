\chapter{Marco Te�rico}

\section{Introducci�n}
El marco te�rico debe contener los fundamentos necesarios para llevar a cabo el desarrollo del proyecto.
Sin caer en un corta-pega y/o una extensi�n excesiva y dando el cr�dito que correspondiente a las fuentes consultadas.
Este cap�tulo puede ser particionado en dos o tres
cap�tulos si los asesores interno y externo deciden que es la mejor opci�n para organizaci�n y
presentaci�n del documento.

El marco te�rico describe las herramientas y t�cnicas de Ingenier�a (Industrial, Sistemas Computacionales, Electromec�nica, Mecatr�nica) que ser�n utilizadas para resolver la situaci�n problem�tica expresada en este trabajo de residencias profesionales.
Al marco te�rico le corresponde la funci�n de orientar y crear las bases te�ricas de la investigaci�n.
El marco te�rico no es un resumen  de las teor�as que se hayan escrito sobre el tema objeto de la investigaci�n.
Es una revisi�n de quienes est�n investigando o han investigado el tema y los planteamientos de estos autores y cu�les son los principales aspectos por ellos estudiados.

De estas herramientas se debe dar:
\begin{itemize}
	\item La definici�n
	\item Conceptos
	\item Historia
	\item Descripci�n de los m�todos, t�cnicas, a emplear en la resoluci�n del problema
	\item Actualidad que existe sobre el tema y resultados que se hallan obtenido, entre otras cosas
  \item Siempre citando las referencias donde se sustenta lo investigado, ej: ...como se ve en \cite{libr1,capi1,revi1}; sin embargo, en \cite{pagi1} se utiliz� ...\\
\end{itemize}
%Cuando se utilice referenciar a las fuentes de informaci�n, se usa \cite{}
%el cual cita usando en el formato IEEE mostrado en "referencias",
%latex da un n�mero automaticamente a las referencias, como se vio en el ultimo item de arriba

\section{Tema 1 investigado}
Los ... como por ejemplo ...

\subsection{Subtema 1 del Tema 1}
La parte primaria ... aqui va lo investigado

\subsection{Subtema 2 del Tema 1}
La parte secundaria ... aqui va lo investigado, y asi sucesivamente


\section{Figuras}
Si se utilizan figuras recordar referenciarlas en el texto,
he indicar una referencia en caso de que sean tomadas de alguna fuente, ejemplo:
TEXTO ... cuya pintura se muestra en la Figura \ref{dragon1} \cite{capi1}.

\begin{figure}[hbt!]
\centering
\includegraphics[width=0.6\textwidth]{figuras/ciruelo1}
\caption{Obra del artista Ciruelo.}
\label{dragon1}
\end{figure}

\section{Tema 2 investigado}
Los ... se definen como ...

\subsection{Subtema 1 del Tema 2}
La parte primaria ... aqui va lo investigado

\subsection{Subtema 2 del Tema 2}
La parte secundaria ... aqui va lo investigado, y asi sucesivamente

\section{Tablas}
Si se utilizan tablas recordar referenciarlas en el texto,
he indicar una referencia en caso de que sean tomadas de alguna fuente.
Al igual que la palabra Figura, la palabra Tabla tambi�n se acostumbra escribirla con mayuscula, ejemplo:
TEXTO ... en la Tabla \ref{Tuno} \cite{tesis1}.
Observe que las citas van antes del punto, otro ejemplo de colocar datos en Recuadros se ve en la Tabla \ref{TarjHD}.

\begin{table}[!htb]
\caption{Tabla a modo de ejemplo.}
\begin{center}
\begin{tabular}{|c|r|l|}							%|c|r|l| significa: linea-centrar-linea-derecha-linea-izquierda-linea
\hline
\multicolumn{3}{|c|}{uni tres columnas}\\ %para esta linea se cambia |c|r|l| por: |c|
\hline
t1&t2&t3\\														%el resto de las lineas sigen el formato |c|r|l|
\hline
0&0&0\\
00&00&00\\
000&000&000\\
0000&0000&0000\\
\hline
\end{tabular}
\end{center}
\label{Tuno}  %con este nombre se referencia en el texto: \ref{Tuno}
\end{table}

\begin{table}[!htb]
\caption{Caracteristicas de tarjeta de audio HD.}
\begin{center}
\begin{tabular}{|c|l|}
\hline
Rango&Caracteristicas\\
\hline
 &Flujo de I/O, compatibilidad con escalabilidad\\
0 - 14 &para el control, enlace (link) y codec dise�ada\\
 &para optimizar el rendimineto y caracteristicas\\
\hline
6K Hz y 192K Hz&Frecuencias de muestreo\\
\hline
xxxx&yyy\\
\hline
 &....\\
48 Mbps& ...\\
 &...\\
\hline
\end{tabular}
\end{center}
\label{TarjHD} %con este nombre se referencia en el texto: \ref{TarjHD}
\end{table}

%%%%%%%%%%%%%%%% termina el capitulo
%\sigpag