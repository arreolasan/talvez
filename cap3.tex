\chapter{Desarrollo y resultados}

\section{Metodo propuesto}
En este cap�tulo se describen las actividades realizadas para cumplir con los objetivos espec�ficos propuestos.
Se discuten los resultados obtenidos respecto al objetivo general ilustrando mediante diagramas;
tablas; modelos matem�ticos; teoremas; series de tiempo, planos y espacios de fase; secciones de c�digo fuente, im�genes de interfaces gr�ficas;
pseudoc�digos; im�genes de ayudas visuales; procedimientos; fotograf�as de prototipos, estaciones de trabajo, fixturas, etiquetas, etc.

Explicar de una manera clara como se utilizaron cada una de las herramientas de Ingenier�a (Industrial, Sistemas Computacionales, Electromec�nica, Mecatr�nica), descritas en el marco te�rico, para llevar a cabo los planteamientos a posibles soluciones de problema que se estudia.

\section{Instrumentos empleados}
De haberse dado se deber�n anexar:\\
Los formatos que se establecieron para un mejor control de las actividades que se desarrollan.\\
Planos, gr�ficas y/o diagramas que auxilien para una descripci�n clara de las propuestas hechas para la soluci�n del problema.\\
Los c�lculos realizados para la soluci�n del problema.

\section{Resultados}
Especificar cuales son los resultados obtenidos para aplicar las herramientas de Ingenier�a (Industrial, Sistemas Computacionales, Electromec�nica, Mecatr�nica).

El prototipo dise�ado se muestra en la figura ... etc.



%%%%%%%%%%%%%%%%
\sigpag